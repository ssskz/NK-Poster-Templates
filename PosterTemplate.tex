\documentclass[final]{beamer}

% ====================
% Packages
% ====================
\usepackage[T1]{fontenc}
\usepackage{lmodern}
\usepackage[orientation=portrait,size=a0,scale=1.0]{beamerposter} % Current dimensions A0, put in your poster dimensions
\usetheme{gemini}
\usecolortheme{uchicago}
\usepackage{graphicx}
\usepackage{caption}
\usepackage{booktabs}
\usepackage{tikz}
\usepackage{pgfplots}
\pgfplotsset{compat=1.17}
\newcommand{\blu}{\color{blue}}
\usepackage{DejaVuSans}
%%%%%%%%%%%%%%%%%%%%%%%%%%%%%%%%%%%%%%%%%%%%%%%%%%%%%%%%%%%%%%%%%%%%%%%%%%%%%%
% Column environment setup
%%%%%%%%%%%%%%%%%%%%%%%%%%%%%%%%%%%%%%%%%%%%%%%%%%%%%%%%%%%%%%%%%%%%%%%%%%%%%%
% If you have N columns, choose \sepwidth and \colwidth such that
% (N+1)*\sepwidth + N*\colwidth = \paperwidth
% Follow structure to create difference column environments. 
\newlength{\sepwidthA} % Seperation distance between comulumns type A
\newlength{\colwidthA} % collumn width type A
\setlength{\sepwidthA}{0.25\paperwidth}
\setlength{\colwidthA}{0.5\paperwidth}

\newcommand{\separatorcolumnA}{\begin{column}{\sepwidthA}\end{column}}

% Second column environment. 
\newlength{\sepwidthB}
\newlength{\colwidthB}
\setlength{\sepwidthB}{0.0666\paperwidth}
\setlength{\colwidthB}{0.4\paperwidth}

\newcommand{\separatorcolumnB}{\begin{column}{\sepwidthB}\end{column}}

% You can also use these column commands to create columns inside columns and for creating new column formatting. 
% You can also have non even columns by creating more column environments or specifying the width when beginning a column environment. 
%%%%%%%%%%%%%%%%%%%%%%%%%%%%%%%%%%%%%%%%%%%%%%%%%%%%%%%%%%%%%%%%%%%%%%%%%%%%%%
% Title
%%%%%%%%%%%%%%%%%%%%%%%%%%%%%%%%%%%%%%%%%%%%%%%%%%%%%%%%%%%%%%%%%%%%%%%%%%%%%%

\title{\VeryHuge{Title}}
\author{\textbf{Shengquan Chen}, Kezheng Su and Author}
\institute[shortinst]{Institute of Mathematical Sciences, Nankai University, Tianjin}

%%%%%%%%%%%%%%%%%%%%%%%%%%%%%%%%%%%%%%%%%%%%%%%%%%%%%%%%%%%%%%%%%%%%%%%%%%%%%%
% Poster footer
%%%%%%%%%%%%%%%%%%%%%%%%%%%%%%%%%%%%%%%%%%%%%%%%%%%%%%%%%%%%%%%%%%%%%%%%%%%%%%

\footercontent{
\href{mailto:chenshengquan@nankai.edu.cn}{chenshengquan@nankai.edu.cn} % this is a clickable link
  \hfill
  National innovation project \hfill
  {Written by Kezheng $\text{Su}^{\copyright}$ } }
% (can be left out to remove footer)  

\begin{document}
%%%%%%%%%%%%%%%%%%%%%%%%%%%%%%%%%%%%%%%%%%%%%%%%%%%%%%%%%%%%%%%%%%%%%%%%%%%%%%
% Logo placements (optional)
%%%%%%%%%%%%%%%%%%%%%%%%%%%%%%%%%%%%%%%%%%%%%%%%%%%%%%%%%%%%%%%%%%%%%%%%%%%%%%
\addtobeamertemplate{headline}{}
{
    %\begin{tikzpicture}[remember picture,overlay] % Solid header bar
    \begin{tikzpicture}[remember picture,overlay,line width=\arrayrulewidth] % gradient header bar
      % UQ Reverse Logo 
      \node [anchor=north west, inner sep=3cm] at ([xshift=0cm,yshift=1.0cm]current page.north west)
      {\includegraphics[height=7.0cm]{logos/5}}; 
      % Logo 1, replace with custom logo
      \node [anchor=north east, inner sep=3cm] at ([xshift=0.0cm,yshift=2.0cm]current page.north east)
      {\includegraphics[height=8.0cm]{logos/Logo-Right.png}}; 
      % Extra logo 2
      \node [anchor=north east, inner sep=3cm] at ([xshift=1.0cm,yshift=-10.0cm]current page.north east)
      {\includegraphics[height=6.0cm]{logos/nk_logo}};
      %  Extra logo 3, or a QR Code
      \node [anchor=north west, inner sep=3cm] at ([xshift=-0.5cm,yshift=-9.5cm]current page.north west)
      {\includegraphics[height=6.0cm]{Figures/QRcode}}; 
    \end{tikzpicture}
}

% ====================
% Body
% ====================

\begin{frame}[t]
%%%%%%%%%%%%%%%%%%%%%%%%%%%%%%%%%%%%%%%%%%%
%Section 1
%%%%%%%%%%%%%%%%%%%%%%%%%%%%%%%%%%%%%%%%%%%
\begin{columns}[t]
    \separatorcolumnA
    \begin{column}{\colwidthA}

        \begin{block}{Introduce graphical abstract, pose research question}
            \begin{figure}
                \centering
                \includegraphics[width=1.0\textwidth]{logos/2}
            \end{figure}
        \end{block}
    
    \end{column}

    \separatorcolumnA
\end{columns}

%%%%%%%%%%%%%%%%%%%%%%%%%%%%%%%%%%%%%%%%%%%
%Section 2 
%%%%%%%%%%%%%%%%%%%%%%%%%%%%%%%%%%%%%%%%%%%
\begin{columns}
%%%%%%%%%%%%%%%%%%%%%%%%%%%%%%%%%%%%%%%%%%%
%Section 2 column 1
%%%%%%%%%%%%%%%%%%%%%%%%%%%%%%%%%%%%%%%%%%%
\separatorcolumnB

    \begin{column}[T]{\colwidthB}

        \begin{block}{Block with a diagram/figure}

        Here is a diagram of a star drawn in tikz.

        \begin{figure}
            \centering
                \begin{tikzpicture}[scale=8]
                \draw[step=0.25cm,color=gray] (-1,-1) grid (1,1);
                \draw (1,0) -- (0.2,0.2) -- (0,1) -- (-0.2,0.2) -- (-1,0)
                -- (-0.2,-0.2) -- (0,-1) -- (0.2,-0.2) -- cycle;
                \end{tikzpicture}
            \caption{A figure caption. If needed for data or reference}
        \end{figure}
    \end{block}
    
    \begin{block}{Block with nested columns}
        \begin{column}{0.3\colwidthB}
            \begin{itemize}
                \item Keep text information clear and concise
                \item You can usse some math as well: $ F = ma $
            \end{itemize}
        \end{column}
        \begin{column}{0.7\colwidthB}
            \begin{figure}
            \centering
                \includegraphics[width=0.9\textwidth]{logos/1}
            \end{figure}
        \end{column}
    \end{block}

    \begin{alertblock}{Highlighted Block}
        My boss, Shengquan Chen, is now looking for motivated graduate/undergraduate students. If you are interested in and want to collaborate with my boss!
    \end{alertblock}
        
\end{column}
\separatorcolumnB
%%%%%%%%%%%%%%%%%%%%%%%%%%%%%%%%%%%%%%%%%%%
%Section 2 column 2
%%%%%%%%%%%%%%%%%%%%%%%%%%%%%%%%%%%%%%%%%%%
\begin{column}[T]{\colwidthB}

    \begin{block}{Block With a Table}
    A table in a block for torturing your audience. 
        \begin{table}
            \centering
                \begin{tabular}{c c c c}
                \toprule
                \textbf{ Colour } & \textbf{ Red } & \textbf{ Green } & \textbf{ Blue } \\
                \midrule
                NK Purple & 126 & 12 & 110 \\
                White & 255 & 255 & 255 \\
                Black & 0 & 0 & 0 \\
                \bottomrule
            \end{tabular}
            \caption{A table caption.}
        \end{table}
    \end{block}

  

    \begin{block}{QR code for linking online resources}
        \begin{figure}
            \centering
            \includegraphics[width=1\textwidth]{Figures/chen}
        \end{figure}
\begin{column}[T]{0.6\colwidthB}
      This QR code goes to author's wechat account.If you have any suggestions,you are welcome to consult,or you can visit my Github pages to commit your issues.
\end{column}
\begin{column}[T]{0.4\colwidthB}
    \begin{figure}
      \centering
      \includegraphics[width=0.5\textwidth]{Figures/QRcode}
      \captionsetup{labelformat=empty}
        \caption{\href{https://github.com/ssskz}{\textbf{\blu{Clickable link to URL}}}}
     \end{figure}
\end{column}
\end{block}

%%%%%%%%%%%%%%%%%%%%%%%%%%%%%%%%%%%%%%%%%%%%%%%%%%%%%%%%%%%%%%%%%%%%%%%%%%%%%%
% References
%%%%%%%%%%%%%%%%%%%%%%%%%%%%%%%%%%%%%%%%%%%%%%%%%%%%%%%%%%%%%%%%%%%%%%%%%%%%%%
\begin{block}{References}

\nocite{*}
\bibliography{poster}% Produces the bibliography via BibTeX.
\bibliographystyle{plain}
\end{block} 

\end{column}
\separatorcolumnB
\end{columns}
\end{frame}
\end{document}
